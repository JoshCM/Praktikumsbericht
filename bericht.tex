\documentclass[11pt]{article} % ein Artikel in 11-Punkt Schrift
% wie man sich schon denkt leitet % einen Kommentar bis Zeilenende ein

\usepackage[german]{babel} % deutsch, deutsche Rechtschreibung
\usepackage[utf8]{inputenc} % Unicode Text 
\usepackage[T1]{fontenc} % Umlaute und deutsches trennen
\usepackage{mathptmx} % Times New Roman, gewohnter Font
\usepackage{courier} % Schreibmaschinenfont schicker
\usepackage[scaled=.95]{helvet} % was serifenloses wenn gebraucht
\usepackage{graphicx} % wir wollen Bilder einfügen

\usepackage{listings} % Schöne Quellcode-Listings
\lstset{basicstyle=\sffamily, columns=[l]flexible, mathescape=true, 
  showstringspaces=false, numbers=left, numberstyle=\tiny}
\lstset{language=python} % und nur schöne Programmiersprachen ;-)
% und eine eigene Umgebung für Listings
\usepackage{float}
\newfloat{listing}{htbp}{scl}[section]
\floatname{listing}{Listing}

% Auch wenn es anrüchig ist, man kann den Platz etwas mehr ausnützen
\usepackage[paper=a4paper,width=14cm,left=35mm,height=22cm]{geometry}
\usepackage{setspace}
\linespread{1.5} % nicht ganz anderthalbzeilig, nur ein bisschen mehr Platz
\setlength{\parskip}{0.5em} % kleiner Paragraphenabstand
\setlength{\parindent}{0em} % im Deutschen Einrückung nicht üblich, leider

% Seitenmarkierungen 
\usepackage{fancyhdr} % Schickere Header und Footer
\pagestyle{fancy}
% Zeichensatz für Header/Footer
\newcommand{\phv}{\fontfamily{phv}\fontseries{m}\fontsize{9}{11}\selectfont}
\fancyhead[L]{\phv Praktikumsbericht} 
\fancyhead[R]{\phv \thepage}
\fancyfoot[L]{\phv Hochschule RheinMain}
\fancyfoot[C]{\ } % keine Seitenzahl unten
\fancyfoot[R]{\phv Medieninformatik}

% Ein spezielles Paket zum Aufteilen des Literaturverzeichnisses
\usepackage{bibtopic}
\usepackage{url} % wir wollen eine URL anzeigen

\title{Praktikumsbericht}
\author{Joshua Coelho Mestre}
\date{\today} % oder \today für heute

\begin{document}

\maketitle
\begin{abstract}
Der Bericht zum absolvierten Pflichtpraktikum bei Scholz \& Volkmer Wiesbaden.
\end{abstract}
\newpage
\tableofcontents % das Inhaltsverzeichnis
\newpage % neue Seite, muss bei einem Artikel eigentlich nicht sein

\section{Einleitung} \label{sec:einf}
Scholz \& Volkmer ist eine Kreativagentur zur digitalen Markenführung mit ungefähr 130 Mitarbeitern und Firmensitz in Wiesbaden und Berlin.

Die Agentur arbeitet im Kundenauftrag und erstellt Websites, Mobile Apps und weiter digitale Produkte zur Verbesserung der digitalen Präsenz.

In meiner Zeit bei Schovo arbeitete ich vorwiegend als Frontend Entwickler für den Kunden Adidas, der neben seiner Corporate Seite auch einen Blog bei Schovo entwickeln lässt.

Während meines Praktikums war ich für die Umsetzung eines umfangreichen A/B-Tests der auf dem oben erwähnten Blog namens "Gameplan A" durchgeführt werden sollte.


\newpage

\section{Kalenderwoche 23} \label{sec:kw23}

Das Praktikum begann mit einem großen Meeting bei dem ich die Gelegenheit bekam mich der ganzen Firma vorzustellen.

Darauf folgte der Erhalt meines Arbeitsrechners in der IT-Abteilung sowie das treffen mit meiner Praktikumsbetreuerin.

Meine Praktikumsbetreuerin zeigte mir meinen Arbeitsplatz und gab mir eine Einführung in die firmeneigene Softwarestruktur und Abläufe. Darunter zum Beispiel den zum Dateien-Austausch genutzten Fileserver, das häufig genutzte Ticketsystem Jira und das für den Informationsaustausch genutzte Confluence.

Ich machte mich danach mit den eben genannten Systemen vertraut und richtete meinen Arbeitsrechner ein.

Am folgenden Tag durfte ich mich mit dem Code der firmeneigenen Website vertraut machen. Dazu klonte ich das, das sich auf dem Gitlab-Server befindliche Projekt, und lies es mittels Docker-Container auf meinem Arbeitsrechner laufen.

Das Setup des Projekts erfolgte mittels im Repository enthaltener Scripte.

Da geplant war mich als Backend-Developer einzusetzen und es sich bei der S-V.de Webseit um ein Django Projekt handelt, absolvierte ich den Rest der Woche ein detailliertes Django Tutorial um meine im Studium erhaltenen Django-Kenntnisse aufzufrischen und zu verbessern.

Das Ziel des Tutorials war es, abgesehen von der Verbesserung meiner Django Kenntnisse, eine Website zu erstellen, die es den Usern erlaubt Links zu posten und diese einer Bestimmten Category zuzuordnen.

Zusätzlich dazu sollte es den Usern ermöglicht werden direkt auf der Seite das Internet nach neuen Seiten anhand von Suchbegriffen zu durchforsten und bereits vorhandene Links zu "liken".

Das Tutorial befasste sich mit allen relevanten Gesichtspunkten der Webentwicklung und vermittelte zusätzlich zu dem Basiswissen der Django-Programmierung auch Inhalte wie die Nutzung von CSS-Frameworks wie Bootstrap und die Nutzung externer APIs um die Funktionalität der Webseite um eine Websuche zu erweitern. 

\section{Kalenderwoche 24} \label{sec:kw24}

Zu Beginn der zweiten Woche beschäftigte ich mich mit einem auf der s-v.de auftretenden Problem.

Im falle einer Unerreichbarkeit der CDN-Server auf denen die Medien der Seite Abgelegt sind sollten diese aus dem Cach des Django Caches geladen werden, was jedoch nicht zuverlässig geschah.

Ich beschäftigte mich zunächst mit der lokalen Reproduktion des Fehler um ihn dann schrittweise zu lokalisieren.
Auf Grund der Projektlage wurde ich jedoch von dem s-v.de-Projekt abgezogen und auf die Adidas Corporate Seite angesetzt.

Der Kunde wünschte sich ein Update für das sich auf der Seite befindliche Medien-Archiv.

Das für die Filterung der Medienformate benutzte Dropdown-Menü, welches es dem Nutzer ermöglicht aus Audio, Video und Bilder zu wählen, sollte durch drei nutzerfreundlichere Checkboxen ersetzt werden.

Nach dem Klonen und lokalen einrichten des Projekts bestand meine Arbeit zunächst daraus mich mit dem Projekt und dem verwendeten, firmeneigenen Javascript Framework, vertraut zu machen.

Nach meiner Orientierung begann ich mit der Umsetzung des Updates, indem ich zunächst das unerwünschte Dropdown-Menü aus dem betroffenen HTML-Template entfernte und das Markup für die geforderten Checkboxen inklusive des, von der Konzepterin geforderten, Stylings hinzufügte.

Der nächste Schritt beinhaltete Implementierung der Filter-Funktionalität der Checkboxen, wobei sich das Hauptproblem darin ergab, dass statt wie im Falle des Dropdown nun nicht mehr nur eine Kategorie Medien ausgewählt werden kann sondern mehrere oder sogar gar keine Checkbox ausgewählt sein kann.

\section{Kalenderwoche 25} \label{sec:kw25}

Um das Problem der Mehrfachfilterung der Medien durch die Checkboxen zu lösen habe ich mich intensiv mit dem bereits vorhandenen Filter-Modul beschäftigt.

Die Anforderung war das bereits existierende Modul mit möglichst geringer Kopplung und möglichst minimal inversiv dahingehend zu verändern, dass es mit der Eingabe über die Checkboxen die gewünschten Medien einblendet. 

Da das Filter-Modul mehrfach auf der Seite benutzt wird und bis dahin ausschließlich über die Dropdown-Menüs gesteuert wurde stand ich vor der Herausforderung die Funktionalität des Moduls um die Checkbox zu erweitern jedoch das verarbeiten der Eingaben durch die Dropdowns nicht zu behindern.

Um der Anforderung gerecht zu werden habe ich innerhalb des Moduls eine Funktion implementiert, die im Falle einer Eingabe durch die Checkboxen aufgerufen wird und diese soweit verarbeitet, dass sie danach von wie gewohnt von dem Filter weiter genutzt werden kann.

Durch die selbst implementierte Funktion musste ich anschließend nur noch geringfügige Anpassungen an dem Modul vornehmen, da das nun nicht mehr nur noch ein Filter aktiv sein kann sondern mehrere.

Die Änderung beschränkte sich auf die Erweiterung des Moduls um die Möglichkeit durch ein Filter-Array zu iterieren statt nur ein einziges Filterwort zu akzeptieren. 

Nachdem ich den Task beendet hatte ließ ich meinen Code von einem Senior Developer reviewen, der mich auf einige Sachen hinwies, die ich nicht bedacht hatte.

Unter anderem hatte ich Funktionen benutzt, die von Internet Explorer 11 nicht unterstützt werden.
Dies ist vor allem problematisch, da viele Rechnerau Seiten des Kunden diesen Browser noch benutzen.

Ich behob die durch die Code-Review entdeckten Fehler und nutzte die Gelegenheit um noch weitere kleine Verbesserungen vorzunehmen, die ich durch mein verbessertes Verständnis für Javascript und das benutzte Framework entdeckt habe.

\section{Kalenderwoche 26} \label{sec:kw26}

Da ich mich nun in der Projekt Adidas Corporate Seite, und insbesondere in das Medien-Archiv, eingearbeitet hatte übernahm ich weitere Tickets diesbezüglich.

Ich kümmerte mich in dieser Woche um einen bekannten Bug, der zur Auswirkung hatte, dass wenn man sich das Medien-Archiv im apple-eigenen Browser, Safari, anschauen wollte, keine Bilder zu sehen waren.

Ich begann damit den Code zu debuggen und mir parallel dazu sowohl die für die Bilder benutzten Templates, als auch das für das laden der Bilder verantwortliche Javascript-Modul zu Gemüte zu führen.

Mir fiel beim Debuggen auf, dass wenn ich das "loaded"-Event für die Bilder händisch abfeuerte, alle Bilder korrekt angezeigt wurden.

Dieses Verhalten wies darauf hin, dass die Bilder bereits alle geladen waren, jedoch auf Grund des fehlenden Events "hidden" blieben. Dies entsprach dem Code den ich im Bilder-Modul vorfand, wo mit einem Eventlistener auf das Laden der Bilder gewartet wurde um dann alle gleichzeitig anzeigen zu können.

Das Laden der Bilder übernimmt ein Lazyloader, welcher das src-Attribut der Images erst setzt, wenn sie benötigt werden, sodass unnötige clientseitige Operationen und Request gespart werden können.

Die Vermutung lag nahe, dass das Event gefeuert wird noch bevor das Bilder-Modul den Eventlistener anmelden kann.

Durch weiteres Debugging fiel schlussendlich auf, dass bereits im Template das src-Attribut gesetzt wurde anstatt die Quelle in das data-src-Attribut zu schreiben, was zum laden der Bilder führte noch bevor das Javascript initialisiert war wodurch nicht auf das Event reagiert werden konnte, da es nicht gefeuert wurde, da durch das bereits gesetzte Attribut der Lazyloader übergangen wurde.

Eine Korrektur des Fehlers im Template eliminierte den Bug und ermöglichte eine Korrekte anzeige in allen gängigen Browsern.

\section{Kalenderwoche 27} \label{sec:kw27}

In der Kalenderwoche 27 begann meine Arbeit an dem Projekt Gameplan A. Dabei handelt es sich um den Blog von Adidas der sich an sogenannte Business-Athleten richten soll.

Ich wurde mit der Aufgabe Betraut drei Umfangreiche A/B-Tests auf der Seite zu implementieren, die dazu dienen sollen den Blog, der auch zur Akquise von jungen aufstrebenden Mitarbeitern für Adidas, für Nutzer attraktiver zu gestallten.

Zu beginn der Woche erhielt ich ein Briefing zu den geplanten A/B-Tests durch das Projektmanagement und das Konzept.

Der erste umzusetzende Test beschäftigte sich mit der Tonalität des Blogs. Die These die es zu überprüfen galt ist, dass eine andere Formulierung der Headlines zu mehr Pageviews pro Session führt.

Die Anforderung lautete einen A/B-Test mit Hilfe von Google Optimize zu entwickeln, der diese These entweder bestätigt oder verwirft.

Google Optimize ist ein Tool welches erlaubt A/B-Tests durchzuführen und dabei auch andere Google Tools wie den Tagmanager und/oder Google Analytics einzubinden. Dadurch eigente sich Optimize besonders gut für die Verwendung auf Gameplan A, da beide genannten Tools verwendet werden.

Optimize ermöglicht durch einen eigenen Editor zusätzliches CSS und/oder Javascript für eine Variante der Seite zu injizieren.

Da es darum ging auf der gesamten Seite alle Headlines der Blog-Posts zu ändern und dies über Javascript sehr rechenintensiv gewesen wäre entschieden wir uns für eine Implementierung über das CSS der Seite.

Wir nutzten die Möglichkeiten von Optimize und setzten bei der Variante der Seite eine CSS-Klasse im Body-Tag namens "gpa-test-ab-headline-X" wodurch wir mit zusätzlichem CSS die wir dann bereits mit der Seite an den Client liefern auf die Variante reagieren können und alle Headlines des Blogs austauschen können.

Am Ende der Woche besuchte ich eine SEO Schulung welche von einem SEO Experten gehalten wurde, der seitens der Agentur engagiert wurde um den Kunden eine bessere Performance bei Suchmaschinen bieten zu können.

Die Schulung erstreckte sich über einen gesamten Tag, wobei sich der Vormittag einem Überblick über das Thema SEO, und der konzeptionellen
Aspekte des Themas widmete. 

Der Nachmittag befasste sich dann mit allen technischen Aspekten von SEO, was für mich und meine Abteilungs-Kollegen eine besonders hohe Relevanz hatte und sehr gut aufbereitet wurde.

\section{Kalenderwoche 28} \label{sec:kw28}

In dieser Woche befasste ich mich zunächst mit der Backend-Seitigen Anpassungen, die für den A/B-Test nötig waren.

Bei dem Backend handelt es sich bei Gameplan A um eine auf Kundenbedürnisse angepasste Version von Wordpress, die mit einem komplett eigenentwickeltem Theme operiert.

Um ein Austauschen der Titel zu realisieren müssen die Editoren des Blogs zunächst die Möglichkeit haben den einzelnen Blog-Posts einen alternativen Titel zu geben und diesen für den jeweiligen Post zu speichern um den alternativen Titel später an den Client zu liefern.

Ich habe dafür das Blog-Post Formular, mit dem Posts erstellt und bearbeitet werden, im Backend um das Feld "Alternative Headlines" erweitert und die dort hinterlegten Titel dann in den Metadaten des Post gespeichert.

Dies erforderte ein eigenes php-Modul zur Verarbeitung und Speicherung der Daten und die Erstellung eines neues Twig-Templates zur Erstellung der zusätzlichen Feldes.

Die Twig-Templating-Sprache wird im Gameplan A Projekt benutzt um eine klare Trennung des php-Codes und des Markups zu erhalten.

Neben meiner Tätigkeit für Gameplan A wurde ich von einem Projektmanager, der sich primär um den Kunden Fraport kümmert um eine Technische Konsultation gebeten. Dabei ging es darum, dass der Kunde in Besitz neuer Geo-Json Daten des Flughafen Frankfurt gelangt ist und nun die Frage aufkam was genau diese Daten repräsentieren und welche Möglichkeiten zur Visualisierung dieser Daten es gibt.

Da ich bereits in einem Projekt im Fach "Mobile Computing" in Kontakt mit Geo-Json-Daten gekommen bin konnte ich Fragen bezüglich der Daten zügig und zur Zufriedenheit des Kunden in einem Meeting klären.

Für den A/B-Test habe ich mich nach den Ergänzungen im Backend Angefangen die gespeicherten Titel überall da im Template mitzugeben, wo der Titel eines Blogs auftaucht. Ich habe es in ein Attribut namens "data-test-ab-headline-X" des entsprechenden HTML-Tags geschrieben um mittels der CSS-Funktion attr() darauf zugreifen zu können.

Um den Titel schlussendlich mittels CSS auszutauschen habe ich in einem zusätzlichen .scss-File die Schriftgröße des Originalen Titels auf Null gesetzt und dann mittels einer Pseudo-Klasse den Content der Headline mit dem Wert des oben genannten Attributes gesetzt. 

Das oben beschriebene Austauschen findet selbstverständlich nur Anwendung wenn durch Optimize die in Kapitel \ref{sec:kw27} beschriebene Klasse gesetzt wird. Passiert dies nicht hat der Nutzer das Original erhalten und sieht dementsprechend auch die Originalen Headlines.

\section{Kalenderwoche 29} \label{sec:kw29}



\section{Kalenderwoche 30} \label{sec:kw3}

Zum Abschluss der Woche kehrte ich auf Kundenwunsch zurück zur Adidas Corporate Seite zurück um zusätzlich entstandene Anforderungen umzusetzen.

Der Kunde wünschte das die von mir Implementierten Checkboxen initial alle gecheckt sein sollten und um ihre Medien auch teilen zu können wünschten sie die Implementierung von sogenannten Deeplinks.

Diese Deeplinks ermöglichen das teilen einer Seite wobei der Empfänger durch in der Url codierte Informationen genau den Zustand einer Seite ansehen kann, den der Absender ihm zukommen lassen möchte.

Diese zusätzlichen Informationen in der Url sind nötig, da eine Seite nicht bei jeder Interaktion neu geladen wird sondern durch Javascript verändert wird. Teilt man nun den Link ohne zusätzliche Informationen würde man nur die Seite in ihrem initialenZustand erhalte, was unter Umständen nicht gewünscht ist.

Die Interaktion wird dann durch ein Deeplink-Modul, welches die Informationen aus der Url liest virtuell wiederholt.


\section{Kalenderwoche 31} \label{sec:kw3}
\section{Kalenderwoche 32} \label{sec:kw3}
\section{Kalenderwoche 34} \label{sec:kw3}
\section{Fazit} \label{sec:faz}

\newpage


\newpage

% Listen, wenn überhaupt!, bitte ans Ende und nicht an den Anfang
\listoffigures % Liste der Abbildungen 
\listoftables % Liste der Tabellen
% Als letztes noch das Literaturverzeichnis
\bibliographystyle{plainurl} % mit url
% so wäre es ganz einfach!
%\bibliography{ausarb,online}
% dann mit "bibtex ausarb" bibtexen und das Literaturverzeichnis ist da
% z.B. mit bibtopic kann man die Quellen sauber trennen
\begin{btSect}{ausarb}
\section*{Literaturverzeichnis}
\btPrintCited
\end{btSect}
\begin{btSect}{online}
\section*{Online-Quellen}
\btPrintCited
\end{btSect}
% dann mit "bibtex ausarb1" und "bibtex ausarb2" arbeiten.
% Wir verwenden ausarb<i> weil die Dokumenten-Datei ausarb.tex ist.

\end{document}
